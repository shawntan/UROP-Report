\documentclass[a4paper,12pt]{nurop}

\begin{document}

\title{User-centric Web Information Extraction}

\author{\large{Shawn Tan}\footnote{Student} and \large{Kan Min-Yen}\footnote{Supervisor}\\
	\normalsize\textit{Department of Computer Science,\\
	School of Computing,\\
	National University of Singapore} 
}
\maketitle

\begin{abstract}
	In this report, we study a problem and design an efficient algorithm
	to solve the problem.  We implemented the algorithm and evaluated
	its performance against previous proposed algorithms that solves the
	same problem.  Our results show that our algorithm runs faster.
\end{abstract}

\section{Introduction}
Many problems exist in computer science.  In this project, we 
studied one particular important problem and propose a solution 
for it.  

\subsection{Background}
In this subsection, we briefly discuss the history and background
of the problem.  A detail literature survey is presented in 
Chapter \ref{ch:related}.

The problem we study in this report is an important one.
This problem is first proposed in 1990 in the context
of graph theory \cite{smith90graph}.  Zhang gives the
first algorithm to the problem and applied it to solve several 
problems in artificial intelligence \cite{zhang91ai,zhang92ai}.  
More recently, a slightly different formulation of the problem
is studied independently \cite{kovsky92diff,ali94diff}.  None of the previous work
uses the technique that we propose in this project.  Thus, we 
believe that our algorithm is novel.

Next, we formally defined the problem.  We adopt
the definition given by Kovsky \cite{kovsky92diff}.

We will now describe briefly our solution to the problem
defined above.

\subsection{Paper Organization}
The rest of this paper is organized as follows.  In Section \ref{ch:related}
we describe some related work.

\section{Related Work}
\label{ch:related}
In this section, we present some previous research result
that is related to our work.  

\section{Problem and Algorithm}
We now formally defined our problem and present our solution.

\section{Evaluation}
To evaluate our solution, we implemented the solution 
using C++, and test it on many randomly generated inputs.
We compare our solution with previously proposed method
We found that our method shows significant improvement
over previous solutions.

\section{Conclusion}
In this paper, we have presented a solution to solve a
problem.  Our solution employ a simple technique, yet
is able to achieve significant improvement over previously
proposed method.

\bibliographystyle{nurop}
\bibliography{UROP}

\end{document}
