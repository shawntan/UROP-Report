\section{Path Traversal Alignment}
An important aspect which describes tags which are visually similar is its traversal path from
the \url{<BODY>} tag. Parents of selected tags with similar tag names or that have similar
attributes are likely to be tags that look visually similar on the page. This also implies that
 the elements are probably items that that the user would also be looking for.

We propose using a modification of the Needleman-Wunsch \cite{Needleman1970} global alignment algorithm for this
purpose. Each traversal path could be seen as a sequence of symbols that need to be aligned
based on their common attributes. The more attribute-values they have in common, the more
likely they are ancestors of elements of the same type. However, when reconstructing the
aligned sequence, one needs to be aware that the symbols are not entirely equivalent, and that
only the common attributes should be retained.

Hence, we make the following modifications to the Needleman-Wunsch algorithm:
\begin{enumerate}
	\item The scoring function used is based on the number of attributes in common
	between the 2 elements.
	\item The method for reconstructing the aligned path is given as a parameter. We will
	describe in the following sections why this is needed.
\end{enumerate}

\begin{figure}[htbp]
\singlespacing
	\begin{algorithm}[H]
	\caption{\textsc{PTAlign}$(P,Q,\lambda)$}
	\begin{algorithmic}[1]
		\FOR{$i \leftarrow 1$ to $|P| + 1$}
			\FOR{$j \leftarrow 1$ to $|Q| + 1$}
				\STATE $M_{i,j} \leftarrow \max(M_{i-1,j},M_{i-1,j-1},M_{i,j-1}) + \textsc{Score}(P_i,Q_j)$
			\ENDFOR	
		\ENDFOR
		\STATE $R \leftarrow []$
		\WHILE{$i > 1$ and $j > 1$}
			\IF{$M_{i,j} - \textsc{Score}(P_i,Q_j) = M_{i-1,j-1}$}
				\STATE $R \leftarrow \lambda(P_i,Q_j) + R$
				\STATE $i,j \leftarrow i-1,i-j$
			\ELSIF{$M_{i,j} - \textsc{Score}(P_i,Q_j) = M_{i-1,j}$}
				\STATE $R \leftarrow * + R$
				\STATE $i,j \leftarrow i-1,j$
			\ELSIF{$M_{i,j} - \textsc{Score}(P_i,Q_j) = M_{i,j-1}$}
				\STATE $R \leftarrow * + R$
				\STATE $i,j \leftarrow i,j-1$
	    	\ENDIF	
		\ENDWHILE
		\RETURN $R$
	\end{algorithmic}
	\end{algorithm}
\caption{The \textit{traversal path alignment} algorithm}
\label{fig:lcas}
\end{figure}

Given the input of 2 arrays $P$ and $Q$ of elements, the scoring function is given as
$\textsc{Score}(P_i,Q_j) = |P_i \cap Q_j|$, where $P_i$ and $Q_j$ are elements (sets of
attributes) along $P$ and $Q$. After the alignment, a new generalised traversal path is
created, $P'$, such that $P'_k = P_i \cap Q_j$, where $P_i$ and $Q_j$ are elements in the
original traversal paths that were aligned. Figure \ref{fig:lcas} gives the pseudocode for the
\textit{Path Traversal Alignment} algorithm (PTA).


When selecting additional elements for generalising the XPath, the $\lambda$ function would be
defined as $\lambda(P_i,Q_j) = P_i \cap Q_j$. This would reconstuct a generalised traversal
path with common elements and their common attributes. Gaps in the path are represented as ``*"
 in the output. During the serialisation of the XPath, groups of these are converted to
 \url{//} to represent any descendant.

This process is a bottom-up approach: Users start with a single element, and as more items are
selected, the items captured by the XPath generated grows. For each element along the selected
item's path, the similarities between the tag's attributes are kept, and the differences
removed. This results in an extraction rule with fewer and fewer constraints as more elements
are selected.

\begin{figure}[htbp]
\centering
\small
	\begin{tabular}{|l l l l l l l l|}
	\hline
	$P$ & \url{div}	&	\url{div}	&	\url{div}	&	\url{ul}	&	\url{li}	&	\url{h3}	&	\url{a} \\\hline
	id		&mainContent&	&	&	&	&	&	 \\
	index	&	4	&	1		&	1	&	3	&	\colorbox{yellow}{1}	&	1	&	1 \\
	class	&		&	content	&	newsl, pagedNewsList	&	newslTop &	&	& \\
	\hline
	\end{tabular} \\
	\begin{tabular}{|l l l l l l l l|}
	\hline
	$Q$ 	& \url{div}	&	\url{div}	&	\url{div}	&	\url{ul}	&	\url{li}	&	\url{h3}	&	\url{a} \\\hline
	id		&mainContent&	&	&	&	&	&	 \\
	index	&	4	&	1		&	1	&	3	&	\colorbox{yellow}{2}	&	1	&	1 \\
	class	&		&	content	&	newsl, pagedNewsList	&	newslTop &	&	& \\
	\hline
	\end{tabular}\\
	\textbf{After PTA:}\\
	\begin{tabular}{|l l l l l l l l|}
	\hline
	Result 	& \url{div}	&	\url{div}	&	\url{div}	&	\url{ul}	&	\url{li}	&	\url{h3}	&	\url{a} \\\hline
	id		&mainContent&	&	&	&	&	&	 \\
	index	&	4	&	1		&	1	&	3	&		&	1	&	1 \\
	class	&		&	content	&	newsl, pagedNewsList	&	newslTop &	&	& \\
	\hline
	\end{tabular}
\caption{Example of fewer constraints after each iteration}
\label{fig:lcsexample}
\end{figure}

Using alignment of the path elements, XPath rules can be generalised for elements at different
levels of the DOM tree. Figure \ref{fig:lcsdiagram} is an example of the array generated when 2
\url{<a>} tags are compared, one at a depth of 5 from the \url{<body>} tag, and another at
depth 3. The \url{<body>} tag is omitted in order to reduce the running time of the algorithm,
as it is common in the paths for all visible elements on a given page. The algorithm is then
able to generalise an XPath: \url{//div[@id=`main']/div//a}. This XPath will include both
selected elements, as well as other anchor tags that the user may be interested in.


\begin{figure}[htbp]
\small
\centering

% The state vector is represented by a blue circle.
% "minimum size" makes sure all circles have the same size
% independently of their contents.
\tikzstyle{state}=[rectangle,
                                    thick,
                                    minimum size=1cm,
                                    draw=blue!80,
                                    fill=blue!20]
% The measurement vector is represented by an orange circle.
\tikzstyle{measurement}=[rectangle,
                                                thick,
                                                minimum size=1cm,
                                                draw=orange!80,
                                                fill=orange!25]

% The control input vector is represented by a purple circle.
\tikzstyle{input}=[rectangle,
                                    thick,
                                    minimum size=1cm,
                                    draw=red!80,
                                    fill=red!20]

% The input, state transition, and measurement matrices
% are represented by gray squares.
% They have a smaller minimal size for aesthetic reasons.
\tikzstyle{matrx}=[rectangle,
                                    thick,
                                    minimum size=1cm,
                                    draw=gray!80,
                                    fill=gray!20]

% The system and measurement noise are represented by yellow
% circles with a "noisy" uneven circumference.
% This requires the TikZ library "decorations.pathmorphing".
\tikzstyle{noise}=[circle,
                                    thick,
                                    minimum size=1.2cm,
                                    draw=yellow!85!black,
                                    fill=yellow!40,
                                    decorate,
                                    decoration={random steps,
                                                            segment length=2pt,
                                                            amplitude=2pt}]

% Everything is drawn on underlying gray rectangles with
% rounded corners.
\tikzstyle{background}=[rectangle,
                                                fill=gray!10,
                                                inner sep=0.15cm,
                                                rounded corners=2mm]

\begin{tikzpicture}[>=latex,text height=1.3ex,text depth=0.23ex]
    % "text height" and "text depth" are required to vertically
    % align the labels with and without indices.
  
  % The various elements are conveniently placed using a matrix:
  \matrix[row sep=0.3cm,column sep=0.4cm] {
    % First line: Control input
    &
        \node (u_1) [input]{\url{div[@id=`main']}}; &
        \node (u_2)   [input]{\url{div[1]}};     &
        \node (u_3) [input]{\url{ul}}; &
        \node (u_4) [input]{\url{li}}; &
        \node (u_5) [input]{\url{a}}; &
        \\
        \node (v_1)	[input]{\url{div[@id=`main']}}; &
        \node (A_1)	[input]{\url{div[@id=`main']}}; &
        \node (A_2)	[noise]{\url{div}};       &
        \node (A_3)	[matrx]{\url{NULL}};	&
        \node (A_4)	[matrx]{\url{NULL}};	&
        \node (A_5)	[matrx]{\url{NULL}};	&
        \\
        \node (v_2)	[input]{\url{div[2]}};     &
        \node (B_1)	[noise]{\url{div}};	&
        \node (B_2)	[noise]{\url{div}};     &
        \node (B_3)	[matrx]{\url{NULL}};	&
        \node (B_4)	[matrx]{\url{NULL}};	&
        \node (B_5)	[matrx]{\url{NULL}};	& 
        \\
        \node (v_3) [input]{\url{a}}; &
        \node (C_1)	[matrx]{\url{NULL}};	&
        \node (C_2)	[matrx]{\url{NULL}};	&
        \node (C_3)	[matrx]{\url{NULL}};	& 
        \node (C_4)	[matrx]{\url{NULL}};	& 
        \node (C_5)	[input]{\url{a}};	& 
        \\
        &
        \node (R_1) [measurement] {\url{div[@id=`main']}}; &
        \node (R_2)   [measurement] {\url{div}};   &
        &
        &
        \node (R_3) [measurement] {\url{a}}; &
        \\
    };
	\path[->]
		(C_5) edge[thick] (B_2)
		(B_2) edge[thick] (A_1)
		;

    \begin{pgfonlayer}{background}
        \node [background,
                    fit=(u_1) (u_5)
                    ] {};
                    
      	\node [background,
                    fit=(v_1) (v_3)
                    ] {};
        \node [background,
                    fit=(A_1) (A_5) (C_1)] {};
        \node [background,
                    fit=(R_1) (R_3),
                    label=left:Result:] {};
    \end{pgfonlayer}
\end{tikzpicture}
\caption{Table generated by the LCAS algorithm}
\label{fig:lcsdiagram}
\end{figure}

It is important to note that serialising the generalised model to its XPath form would result
in loss of information. As such the model in its unserialised form is kept, and used again when
another item on the page is selected. Since this process has an associative property, the order
in which items are selected on the page does not matter, and the user can then select 3 items
or more in order to specify the items he wants.

\subsection{Element rejection}

Once the user has defined a region for extraction, he/she may find that the region includes
items that are irrelevant. Our objective was to make this process part of the selection process as
well. In order to achieve this, we used the same alignment algorithm, with $P$ as the already
generalised path, and $Q$ as the new traversal path of the element that the user wishes to
reject, with $\lambda$ defined as $\lambda(P_i,Q_j) = Q_j - (P_i \cap Q_j)$.


\subsection{Limitations}
There are still several limitations to this method. When $Q_j = P_i \cap Q_j$ for all $i,j$
in the aligned path, that rejection path cannot be used. This usually implies that the
traversal paths of the wanted elements and the rejected elements are indistinguishable using
the considered attributes. In such a case, the user is not allowed to select elements that
cause such an event. Once the first rejection element has been generalised with the PTA
algorithm, additional rejection elements can then be generalised using the first algorithm.

Also, the mentioned method of generalising the XPath is a form of a multiple sequence alignment
, but since it is approached using a sequence of pairwise alignments, the optimal alignment may
 not be achieved. However, the selection scopes of the generated XPaths are generally usable
for scraping.

One final limitation that we intend to overcome in our project is that the XPath generated
in this manner will not work after a layout change that affects the layout of the DOM
tree. In the following section, we will describe how we overcome this problem.