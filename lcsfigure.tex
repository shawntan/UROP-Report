\begin{figure}[htbp]
\small
\centering

% The state vector is represented by a blue circle.
% "minimum size" makes sure all circles have the same size
% independently of their contents.
\tikzstyle{state}=[rectangle,
                                    thick,
                                    minimum size=1cm,
                                    draw=blue!80,
                                    fill=blue!20]
% The measurement vector is represented by an orange circle.
\tikzstyle{measurement}=[rectangle,
                                                thick,
                                                minimum size=1cm,
                                                draw=orange!80,
                                                fill=orange!25]

% The control input vector is represented by a purple circle.
\tikzstyle{input}=[rectangle,
                                    thick,
                                    minimum size=1cm,
                                    draw=red!80,
                                    fill=red!20]

% The input, state transition, and measurement matrices
% are represented by gray squares.
% They have a smaller minimal size for aesthetic reasons.
\tikzstyle{matrx}=[rectangle,
                                    thick,
                                    minimum size=1cm,
                                    draw=gray!80,
                                    fill=gray!20]

% The system and measurement noise are represented by yellow
% circles with a "noisy" uneven circumference.
% This requires the TikZ library "decorations.pathmorphing".
\tikzstyle{noise}=[circle,
                                    thick,
                                    minimum size=1.2cm,
                                    draw=yellow!85!black,
                                    fill=yellow!40,
                                    decorate,
                                    decoration={random steps,
                                                            segment length=2pt,
                                                            amplitude=2pt}]

% Everything is drawn on underlying gray rectangles with
% rounded corners.
\tikzstyle{background}=[rectangle,
                                                fill=gray!10,
                                                inner sep=0.15cm,
                                                rounded corners=2mm]

\begin{tikzpicture}[>=latex,text height=1.3ex,text depth=0.23ex]
    % "text height" and "text depth" are required to vertically
    % align the labels with and without indices.
  
  % The various elements are conveniently placed using a matrix:
  \matrix[row sep=0.3cm,column sep=0.4cm] {
    % First line: Control input
    &
        \node (u_1) [input]{\url{div[@id=`main']}}; &
        \node (u_2)   [input]{\url{div[1]}};     &
        \node (u_3) [input]{\url{ul}}; &
        \node (u_4) [input]{\url{li}}; &
        \node (u_5) [input]{\url{a}}; &
        \\
        \node (v_1)	[input]{\url{div[@id=`main']}}; &
        \node (A_1)	[input]{\url{div[@id=`main']}}; &
        \node (A_2)	[noise]{\url{div}};       &
        \node (A_3)	[matrx]{\url{NULL}};	&
        \node (A_4)	[matrx]{\url{NULL}};	&
        \node (A_5)	[matrx]{\url{NULL}};	&
        \\
        \node (v_2)	[input]{\url{div[2]}};     &
        \node (B_1)	[noise]{\url{div}};	&
        \node (B_2)	[noise]{\url{div}};     &
        \node (B_3)	[matrx]{\url{NULL}};	&
        \node (B_4)	[matrx]{\url{NULL}};	&
        \node (B_5)	[matrx]{\url{NULL}};	& 
        \\
        \node (v_3) [input]{\url{a}}; &
        \node (C_1)	[matrx]{\url{NULL}};	&
        \node (C_2)	[matrx]{\url{NULL}};	&
        \node (C_3)	[matrx]{\url{NULL}};	& 
        \node (C_4)	[matrx]{\url{NULL}};	& 
        \node (C_5)	[input]{\url{a}};	& 
        \\
        &
        \node (R_1) [measurement] {\url{div[@id=`main']}}; &
        \node (R_2)   [measurement] {\url{div}};   &
        &
        &
        \node (R_3) [measurement] {\url{a}}; &
        \\
    };
	\path[->]
		(C_5) edge[thick] (B_2)
		(B_2) edge[thick] (A_1)
		;

    \begin{pgfonlayer}{background}
        \node [background,
                    fit=(u_1) (u_5)
                    ] {};
                    
      	\node [background,
                    fit=(v_1) (v_3)
                    ] {};
        \node [background,
                    fit=(A_1) (A_5) (C_1)] {};
        \node [background,
                    fit=(R_1) (R_3),
                    label=left:Result:] {};
    \end{pgfonlayer}
\end{tikzpicture}
\caption{Table generated by the LCAS algorithm}
\label{fig:lcsdiagram}
\end{figure}