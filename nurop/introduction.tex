\section{Introduction}
Web Information Extraction (Web IE) is extracting data from the web,
which are commonly unstructured or semi-structured into a structured form.
Since many web documents today are increasingly template-based, generated by server scripts,
 web IE is slightly different from IE
from free text. This allows for more accurate methods for IE, and systems have been developed
to extract the differing information from the repetitive raw data.
In IE, procedures that extract selected information from structured or semi-structured
documents like HTML are known as \textit{wrappers}.

While many systems for web IE have been developed over the years, many of
of them are within the reach of
corporations and used for extracting huge amounts of data,
but users who browse the web for leisure do not have access to such resources.
Another drawback of these systems is that the process of getting them up and running usually
involves many hours of labelling and training work. As a result, these systems are inaccessible
to users who do not have the time to do this. 

Our primary goal of user-centric web IE is to offer a simple and easy way for users
to be able to extract data from the web. We have identified two challenges for this goal that we will address in this thesis.
Firstly, the system must allow users to specify the information
they want extracted without requiring them to know how to program.
Secondly, the system must work seamlessly even at the presence of a web site layout
change.

To address these challenges, we propose
grab\textit{smart}, a system with the following features:
	\begin{enumerate}
		\item Provide an intuitive and visual interface for labelling that is platform agnostic.
		grab\textit{smart}'s user interface for specifying information on a web page to extract
		 provides the users immediate visual feedback as to items that will be extracted.
		\item Provide a robust framework for extraction of the selected information using
		machine learning. Users would have their wrappers still work after layout changes, with minimal
		user intervention.
	\end{enumerate}