\chapter{Evaluation}

We have divided the evaluation of our system into 2 important sections. The selection interface
deals with a more subjective topic, since its performance is dependent on how easy users find 
selecting items for extraction is. For this component of the system, we are doing a human
evaluation on the qualitative performance of the system. In the second section, we will describe
the evaluation of the machine learning technique we have used, and how well this performs when
attempting to extract data from a webpage after a layout change.

\section{Human evaluation of Selection Interface}

For the evaluation of this part of the system, a group of 30 students were paid to do a simple 30
minute evaluation of the system. A simple tutorial was given to each of the students, and once
completed, a simple extraction task from the bookdepository.co.uk website was to be carried out.
Lastly, the students were asked to fill out a questionnaire on the following aspects of the
system: Ease of installation of the bookmarklet, the ease of item selection, and the ease with
which they could view the extracted items.
\subsection{Analysis}

\section{Machine Learning evaluation}
 We want to be able to extract the same content from pages which have had a layout change. So our
 , evaluation method involves attempting to extract data after a layout change by training a
 classifier for this one.
 
 We have decided to use old pages from web.internet.org to simulate old pages that have not
 undergone 
 a layout change. We created 5 sets of 10 pages from the pages retrieved from web then trained 
 the classifier on each of these. The test results were then compared to the data extracted if an
 XPath was created manually for each of these sets.
\subsection{Analysis} 