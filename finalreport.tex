\documentclass[urop]{socreport}
\usepackage{fullpage}


\usepackage{url}
\usepackage{amsmath,amsthm,amsfonts}
\usepackage{algorithm,algorithmic}
\usepackage[dvips]{color}
\usepackage{algorithm,algorithmic}
\usepackage{graphicx}


\usepackage{tikz}
\usetikzlibrary{trees}
\usetikzlibrary[positioning]
\usetikzlibrary{calc}
\usetikzlibrary{decorations.pathmorphing}
\usetikzlibrary{fit}
\usetikzlibrary{backgrounds}

\begin{document}
\pagenumbering{roman}
\title{User-centric Web Information Extraction}
\author{Shawn Tan}
\projyear{2010/11}
\projnumber{U096883L}
\advisor{A/P Kan Min-Yen}
\deliverables{
	\item Report: 1 Volume
	\item Source Code: 1 DVD}
\maketitle
\begin{abstract}
In this report, I present an algorithm providing an iterative method for users to generate a 
wrapper visually from a given web page, and also a machine learning solution for extraction of
data after a site layout change. These were implemented as a system, and then evaluated 
both quantitatively and qualitatively. 

\begin{descriptors}
	\item H3.5 Web-based Services
    \item H2.8 Data Mining
	\item I5.2 Pattern Analysis
\end{descriptors}
\begin{keywords}
	bookmarklet,  XPath, alignment, information extraction
\end{keywords}
\begin{implement}
	Java, Ruby 1.9.2, Javascript
\end{implement}
\end{abstract}

\begin{acknowledgement}
	I would like to extend my gratitude to my supervisor, A/P Kan Min-Yen, for his patience
when explaining certain concepts to me, and guidance throughout the project. I am grateful
for this opportunity to work on this UROP project under his care. I would also like to thank
the entire WING group, for their valuable advice, especially Jesse Prabawa, for answering 
so many of my questions. 
\end{acknowledgement}

\listoffigures 
\listoftables
\tableofcontents 

\input introduction.tex
\input relatedwork.tex
\input method.tex
\input implementation.tex
\input evaluation.tex
\input conclusion.tex


\bibliographystyle{socreport}
\bibliography{UROP}
\newpage
\appendix
\input surveyresults.tex
\begin{table}
\centering
\small
\singlespacing
\begin{tabular}{|c|c|c|c|c|c|c|c|c|}
\hline
Batch	&Label	&TP	&FP	&FN	&TN	&	&Precision	&Recall\\
\hline
\input mlevalresults.tex
\hline 
\end{tabular}
\label{tab:mlevalresults}
\caption{Learnt model evaluation results for digg.com}
\end{table}




\end{document}
