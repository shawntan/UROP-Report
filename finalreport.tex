\documentclass[urop]{socreport}
\usepackage{fullpage}


\usepackage{url}
\usepackage{amsmath,amsthm,amsfonts}
\usepackage{algorithm,algorithmic}
\usepackage[dvips]{color}
\usepackage{algorithm,algorithmic}
\usepackage{graphicx}


\usepackage{tikz}
\usetikzlibrary{trees}
\usetikzlibrary[positioning]
\usetikzlibrary{calc}
\usetikzlibrary{decorations.pathmorphing}
\usetikzlibrary{fit}
\usetikzlibrary{backgrounds}

\begin{document}
\pagenumbering{roman}
\title{grab\textit{smart}: A User-centric Web Information Extraction System}
\author{Shawn Tan}
\projyear{2010/11}
\projnumber{U07932}
\advisor{A/P Kan Min-Yen}
\deliverables{
	\item Report: 1 Volume
}
\maketitle
\begin{abstract}
We present grab\textit{smart}, a user-centric web information extraction system that
addresses two challenges of today's web information extraction systems. Firstly, users
without any programming language should be able to easily and simply specify information
they want to extracted. Secondly, the system should be robust enough to work seamlessly
after changes in website layouts. To address these challenges grab\textit{smart}: 1.
provides users with an interactive visual interface for information selection, and 2.
uses a robust machine learning framework that extracts information even after layout changes.

Our interface for graphical selection of data for extraction has been found useful by the surveyed students and
our estimated evaluation of our robust wrapper shows that our system is comparable with the state-of-the-art,
performing at 91.07\% according to their measure of robustness, while their performance
is around 80\%.

\begin{descriptors}
	\item H3.5 Web-based Services
    \item H2.8 Data Mining
	\item I5.2 Pattern Analysis
\end{descriptors}
\begin{keywords}
	bookmarklet,  XPath, alignment, information extraction
\end{keywords}
\begin{implement}
	Java, Ruby 1.9.2, Javascript
\end{implement}
\end{abstract}

\begin{acknowledgement}
	I would like to extend my gratitude to my supervisor, A/P Kan Min-Yen, for his patience
when explaining certain concepts to me, and guidance throughout the project. I am grateful
for this opportunity to work on this UROP project under his care. I would also like to thank
the entire WING group, for their valuable advice, especially Jesse Prabawa, for answering 
so many of my questions. 
\end{acknowledgement}

\listoffigures 
\listoftables
\tableofcontents 

\input introduction.tex
\input relatedwork.tex
\input method.tex
\input implementation.tex
\input evaluation.tex
\input conclusion.tex


\bibliographystyle{socreport}
\bibliography{UROP}
\newpage
\appendix
\chapter{Evaluation}
\section{Bookstore scenario}

Now that you have familiarised yourself with the basic usage of the system,
we can move on to our first scenario.

Book Depository is an online bookstore that provides free shipping to many
countries round the world. This gives it a slight edge over the popular
online bookstore Amazon. It's site is a standard e-commerce site, with standard
capabilities such as browsing and searching, and sorting search results.

Imagine you are interested in up and coming mathematics books, but,
unfortunately, Book Depository does not provide an active feed of
such data as it is keyed into its database.

Use the system to create an extractor that provides a feed that would
supply the needed information by extracting the relevant data from the
e-bookstore. The feed must provide the data in order of latest to oldest.
\newpage
\section{Survey \& Feedback results}
\input surveyresults.tex

\end{document}
